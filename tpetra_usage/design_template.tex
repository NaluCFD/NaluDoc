% $Revision: 1.8 $
% $Date: 2008/06/23 19:04:15 $
\makeindex
\documentclass[11pt]{article}
\makeindex
%\usepackage[nomarkers,nolists]{endfloat}
\usepackage{overcite}
\bibliographystyle{biblio}
\usepackage{epsfig}
\usepackage{color}
\usepackage{hyperref}
\usepackage{xr}

%%% two definitions of \Comment. The first displays them. The
%%% second does not. Uncomment one.
\newcommand{\Comment}[1]{
{\noindent\color{red}\fbox{\em \begin{minipage}[t]{5in}#1
\end{minipage}}\color{black}}}
%\newcommand{\Comment}[1]{}
\externaldocument[U:]{users}
\externaldocument[V:]{verification}

\begin{document}
\section{Design Document Template}
%%%%%%%%%%%%%%%%%%%%%%%%%%%%%%%%%%%%%%%%%%%%%%%%%%%%%%%%%%%%%%%%%%%%%%%%%%%%%
 
\Comment{ The following is a design template to assist in building a design
 document. It lists the critical elements of the design, but is not
 intended to be limiting, i.e. you may want to add a lot more to it than
 is found here. There are many ways to describe these design issues
 including nice things like UML. The format is not that critical, but it
 is critical that the essential information be included.

\centerline{\rule{5in}{1pt}}

\noindent
 Generally these completed templates will be inserted into our
 programmers notes.}

\subsection{Purpose of this Section}
The design document is intended to accomplish the following.
\begin{enumerate}
\item
Capture the ideas behind an implementation.
\item
List reasons and motivations for the implementation decisions.
\item
Provide training for other developers who may want to modify or
interface to the software.
\end{enumerate}

\subsection{Requirements Summary}
\Comment{One paragraph summary of the purpose of this work. It
may certainly reference other documentation such as theory manual.}

\subsection{Key Stakeholders}
\Comment{Who will test and use this software.}

\subsection{User Interface}
\Comment{Will the software require changes to the user input?}

\subsection{Theory}
\Comment{What is the theory behind this development? If it is well
understood, are there good references? What should be added to the theory
manual at completion?}

\subsection{Data Structures}

\Comment{A DETAILED description of the data structures that you will use
to store the program's data.  Describe in detail what data needs
to be stored and how it will be stored (e.g., binary trees, hash
tables, queues, arrays, linked-lists, etc.).  Also explain {\em
why} you chose the data structures that you did.}


\subsection{Class Descriptions}
\Comment{For each class in your design, document the following:

\begin{itemize}
\item
    The name and purpose of the class
\item
    The name and purpose of each data member
\item
    The name and purpose of each method.  For each method, document each of its parameters and its return value.
\end{itemize}

One way to document this information is to create a commented
header file (.H file) for each of your classes.  List the classes
here, and ensure that each of the class headers contains the
information above.} %% end comment

\subsection{Algorithms}

\Comment{A DETAILED description of how your program will work.  Describe
how the objects in your design will work together to implement
the functionality.  Describe the flow of your program from
beginning to end, including initialization, parsing, etc.
Explain the core algorithms of your program, including how
control will flow from one method to another.  Explain how you
will handle the various error conditions.

There may be some other design trade-offs. 

This can be documented in a variety of ways. A flowchart or UML
diagram is often of value.}

\subsubsection{Parallelization Issues}

\subsection{Interfaces}
\Comment{Most of the software design we effect must interact with
other systems. Describe this here in as much detail as is
possible.}

\subsection{Testing Issues}


\subsection{Verification}

\begin{table}[ht]
\centerline{
\begin{tabular}{|c|c|c|c|c|}
\hline
\bf Analytic & \bf Verification & \bf Tested & \bf Parallel & \bf User \\
\bf Reference & \bf Section     &            & \bf Test     & \bf Test \\
\hline
             &                  &            &              &          \\
\hline
\end{tabular}
}
\caption{Verification Summary}
%\label{tbl:blah:status}
\end{table}

\subsection{Post-Mortem. Surprises in the Implementation}

\subsection{References}



%%%%%%%%%%%%%%%%%%%%%%%%%%%%%%%%%%%%%%%%%%%%%%%%%%%%%%%%%%%%%%%%%%%%%%%%%%%%%
\end{document}
  

