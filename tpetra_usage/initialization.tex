%&LaTeX

\chapter{Initialization}
Each linear system assembly includes an initialization phase, where the
degrees of freedom and decomposition are defined using Tpetra::Map, and
connectivities are mapped to a compressed row storage (CRS) graph structure.
The following list describes, with some pseudo-code, the broad sequence used
to create and initialize the Tpetra objects. The code that implements these
steps resides in the Nalu class \code{TpetraLinearSystem}.

\begin{enumerate}
\item Create owned and shared maps, and exporter to communicate between them.
\begin{verbatim}
  ownedRowsMap_ = new Map(ownedGlobalIds, ...);
  sharedNotOwnedRowsMap_ = new Map(sharedNotOwnedGlobalIds, ...);
  exporter_ = new Exporter(sharedNotOwnedRowsMap_,
                           ownedRowsMap_);
\end{verbatim}

\item
  Accumulate lists of connectivities defined by
  element-node connections, etc. Currently these connectivity lists are stored
  in a simple vector-of-vectors structure to represent the 'ragged table' of
  node-to-node connections.

\item
  Using mesh information, determine row-lengths due to local connections and due
  to contributions from neighbor processors (which share rows that we own, and
  will thus contribute column-indices to some of our rows). The STK mesh structure
  can tell us the list of neighboring MPI processors, i.e., the MPI processors
  which have portions of the mesh which connect to the local MPI processor's
  portion of the mesh.

  We then traverse our connectivities and for each shared-but-not-owned entity
  we send column entries to the graph on the MPI processor that owns the
  entity. This enables us to compute exact row-lengths for each row in the
  graph.

  Note that this step, and the remaining initialization steps, are implemented
  in the Nalu method \code{TpetraLinearSystem::finalizeLinearSystem()}.

\item
  Create and fill Kokkos::Views representing the local portion of the
  compressed-sparse-row graphs. These are the \code{rowPointers}
  and \code{columnIndices} arrays that will be passed into the
  \code{Tpetra::CrsGraph::setAllIndices} method later. As an implementation
  detail, during construction we hold these Kokkos::Views in a Nalu structure
  called \code{LocalGraphArrays} and this object has methods for inserting
  entries, etc.

\item Construct column map.
  Fill a list of ids representing the union of all column-entries 
  that will be on the local processor. These ids are ordered as
  follows.
\begin{enumerate}
  \item Owned: indices that the local processor owns.
  \item Shared: indices that the local processor shares but doesn't own.
  \item Remote: sometimes called 'reachable', these are indices that
     are neither owned nor shared, but are connected to shared on
     a neighboring processor. These indices are grouped by which
     processor they come from.
\end{enumerate}

\begin{verbatim}
  totalColsMap_ = new Map(colGlobalIds, ...);
\end{verbatim}

\item Construct Graphs.
\begin{verbatim}
  sharedNotOwnedGraph_ = new Graph(sharedNotOwnedRowsMap_,
                                   totalColsMap_, ...);
  ownedGraph_ = new Graph(ownedRowsMap_,
                          totalColsMap_, ...);

  ownedGraph_->setAllIndices(/* our computed local
                                owned-graph indices */);
  sharedNotOwnedGraph_->setAllIndices(/* our computed local
                                         shared-graph indices */);

  importer = new Import(ownedRowsMap_, non-owned-col-ids, source-pids);

  ownedGraph_->expertStaticFillComplete(ownedRowsMap_,
                            ownedRowsMap_, importer, ...);
  sharedNotOwnedGraph_->expertStaticFillComplete(ownedRowsMap_,
                                     ownedRowsMap_, importer, ...);
\end{verbatim}

\item Construct Matrices
\begin{verbatim}
  ownedMatrix_ = new Matrix(ownedGraph_);
  sharedNotOwnedMatrix_ = new Matrix(sharedNotOwnedGraph_);
\end{verbatim}
\end{enumerate}

