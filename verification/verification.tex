%%%%%%%%%%%%%%%%%%%%%%%%%%%%%%%%%%%%%%%%%%%%%%%%%%%%%%%%%
\section{2D Unsteady Uniform Property: Convecting Decaying Taylor Vortex}
%%%%%%%%%%%%%%%%%%%%%%%%%%%%%%%%%%%%%%%%%%%%%%%%%%%%%%%%%

Verification of first-order and second-order temporal accuracy for the
CVFEM and EBVC formulation in Nalu is performed using the method of manufactured 
solution (MMS) technique. For the unsteady isothermal, uniform laminar physics set,
the exact solution of the convecting, decaying Taylor vortex is used.

\begin{equation}
  u = u_o - cos(\pi(x-u_ot)) sin(\pi(y-v_ot))e^{-2.0\omega t}
\label{advConvTV_u}
\end{equation}

\begin{equation}
  v = v_o + sin(\pi(x-u_ot)) cos(\pi(y-v_ot))e^{-2.0\omega t} 
\label{advConvTV_v}
\end{equation}

\begin{equation}
  p = -\frac{p_o}{4}(cos(2\pi(x-u_ot)) + cos(2\pi(y-v_ot)))e^{-4\omega t}
\label{advConvTV_p}
\end{equation}

In this study, the constants $u_o$, $v_o$, and $p_o$ are all assigned values of $1.0$,
and the viscosity $\mu$ is set to a
constant value of $0.001$. The value of $\omega$ is $\pi^2\mu$. This particular viscosity value 
results in a maximum cell reynolds number of twenty.  

\subsection{Temporal Order Of Accuracy Results}
The temporal order of accuracy for the first order backward Euler and second order BDF2
are outlined in Figure~\ref{fig:fo4thTstep} and Figure~\ref{fig:so4thTstep}. Each of these
simulations used a hybrid factor of zero to ensure pure second order central usage. A
fixed Courant number of two was used for each of the three meshes (100x100, 200x200 and 400x400).
The simulation was run out to 0.2 seconds and $L_2$ error norms were computed. The standard
fourth order pressure stabilization scheme with time step scaling is used. This scheme is also
known as the standard incremental pressure, approximate pressure projection scheme.

Two other pressure projection schemes have been evaluated in this study. Each represent a 
simplification of the standard pressure projection scheme. Figure~\ref{fig:hybridTstep} outlines
three projection schemes: the first is when the projected nodal gradient appearing in the fourth-order pressure stabilization is lagged while the second is the classic pressure-free pressure approximate projection scheme with second order pressure stabilization. The third is the baseline fourth-order incremental pressure projection scheme. The error plots demonstrate that lagging the projected nodal gradient for pressure retains second order accuracy. However, as expected the pressure
free pressure projection scheme is confirmed to be first order accurate given the first order splitting error noted in this fully implicit momentum solve.

The Steady Taylor Vortex will be used to verify the spatial accuracy for the full set of advection
operators supported in Nalu.
 
\begin{figure}
\centerline{\includegraphics[width=0.8\textwidth]{figures/convTaylorVortexFO.pdf}}
\caption{Error norms as a function of timestep size for the $u$ and $v$
component of velocity using fourth order pressure stabilization with timestep scaling, backward Euler}
\label{fig:fo4thTstep}
\end{figure}

\begin{figure}
\centerline{\includegraphics[width=0.8\textwidth]{figures/convTaylorVortexSO.pdf}}
\caption{Error norms as a function of timestep size for the $u$ and $v$
component of velocity using fourth order pressure stabilization with timestep scaling, BDF2}
\label{fig:so4thTstep}
\end{figure}

\begin{figure}
\centerline{\includegraphics[width=0.8\textwidth]{figures/convTaylorVortexSO_ElemLagElemPf.pdf}}
\caption{Error norms as a function of timestep size for the $u$ and $v$
component of velocity using the lagged projected nodal pressure gradient and pressure-free pressure projection scheme; all with with timestep scaling, BDF2}
\label{fig:hybridTstep}
\end{figure}

\section{Higher Order 2D Steady Uniform Property: Taylor Vortex}

A higher order unstructured CVFEM method has been developed by Domino~\cite{Domino:2014}. 
A 2D structured mesh study demonstrating second order time and third order in space scheme 
has been demonstrated. The below work has emphasis on unstructured meshes.

\subsection{Source Term Quadrature}
Higher order accuracy is only demonstrated on solutions with source terms when a fully integrated
approach is used. Lumping the source term evaluation is a second order error and is fully noted in
the MMS study (not shown).

\subsection{Projected nodal gradients}
Results show that one must use design order projected nodal gradients. Figure~\ref{fig:pngTempMMS} demonstrates 
a code verification result for a steady thermal manufactured solution comparing lumped and consistent mass matrix approaches for the projected nodal gradient on a quadratic tquad mesh. In the lumped approach, a simple explicit algorithm is processed while for the consistent approach, a simple mass matrix inversion equation must be solved. The lumped approach is first order while the consistent approach retains the expected second order as the projected nodal gradient is expected to be order $P$. Both Dirichlet and periodic domains display the same order of convergence.

\begin{figure}
\centerline{\includegraphics[width=0.8\textwidth]{figures/ho_heatCondMMM_dtdx.pdf}}
\caption{Error norms as a function of mesh size for a CMM and LMM projected nodal gradient on a quadratic tquad mesh.}
\label{fig:pngTempMMS}
\end{figure}

\subsection{Momentum and Pressure}
The steady taylor vortex exact solution was run on a quadratic tquad mesh. Figure~\ref{fig:hoSTVMMS} demonstrates the order of accuracy for projected nodal gradients (pressure) and the velocity field (x-component). Second order accuracy for the projected nodal gradient (pressure) and third order for the velocity field is realized when the consistent mass matrix approach is used for the projected nodal pressure gradient. Note that this term is used in the pressure stabilization approach. However, order of convergence for the projected nodal pressure gradient and velocity field is compromised when the lumped mass matrix approach is used for the pressure stabilization term. Note that both approaches use the fully integrated pressure gradient term in the momentum equation (i.e., $\int p n_i dS$). Therefore, the reduced order of integration for the projected nodal pressure gradient has consequence on the velocity field order of convergence.
 
Again, dirichlet (inflow) and periodic domains display the same order of convergence.

\begin{figure}
\centerline{\includegraphics[width=0.8\textwidth]{figures/ho_stvUandDpDx.pdf}}
\caption{Error norms as a function of mesh size for the Steady Taylor Vortex momentum and pressure gradient field.}
\label{fig:hoSTVMMS}
\end{figure}

%%%%%%%%%%%%%%%%%%%%%%%%%%%%%%%%%%%%%%%%%%%%%%%%%%%%%%%%%
\section{3D Steady Non-isothermal with Buoyancy}
%%%%%%%%%%%%%%%%%%%%%%%%%%%%%%%%%%%%%%%%%%%%%%%%%%%%%%%%%

Building from the basic functional form of the Taylor Vortex,
a non-isothermal solution (momentum, pressure and static enthalpy)
is manufactured as follows:

\begin{eqnarray}
  u &=& -u_o cos(a \pi x) sin(a \pi y ) sin(a \pi z) \nonumber \\
  v &=& +v_o sin(a \pi x) cos(a \pi y ) sin(a \pi z) \nonumber \\
  w &=& -w_o sin(a \pi x) sin(a \pi y ) cos(a \pi z) \nonumber \\
  p &=& -\frac{p_o}{4}( cos(2 a \pi x) + cos(2 a \pi y ) + cos(2 a \pi z) )  \nonumber \\
  h &=& +h_o cos(a_h \pi x) cos(a_h \pi y ) cos(a_h \pi z)  \nonumber \\
\label{3dNonIso}
\end{eqnarray}
%
The equation of state is simply the ideal gas law, 
\begin{equation}
  \rho = \frac{P^{ref} M}{R T}
\label{idealGasEOS}
\end{equation}

The simulation is run on a three-dimensional domain ranging from -0.05:+0.05 with
constants $a, a_h, M, R, C_p, P^{ref}, T_{ref}, Pr, \mu$ equal to 
(20, 10, 30, 10, 0.01, 100, 300, 0.8, 0.00125), respectively.

At reference conditions, the density is unity. The effects of buoyancy are also provided 
by an arbitrary gravity vector of magnitude of approximately ten, $g_i = (-5, 6, 7)^T$. On this domain, 
the enthalpy ranges from zero to unity. Given the reference values, the temperature 
field ranges from 300K to 400K which is designed to mimic a current LES non-isothermal 
validation suite.

Edge- and element-based discretization (P=1) demonstrate second order convergence
in the $L_2$ norm for u, v, w and temperature. This test is captured within the 
variableDensityMMS regression test suite.

%%%%%%%%%%%%%%%%%%%%%%%%%%%%%%%%%%%%%%%%%%%%%%%%%%%%%%%%%
\section{3D Steady Non-uniform with Buoyancy}
%%%%%%%%%%%%%%%%%%%%%%%%%%%%%%%%%%%%%%%%%%%%%%%%%%%%%%%%%

Building from the basic functional form of the Taylor Vortex,
a non-uniform solution (momentum, pressure and mixture fraction)
is manufactured as follows:

\begin{eqnarray}
  u &=& -u_o cos(a \pi x) sin(a \pi y ) sin(a \pi z) \nonumber \\
  v &=& +v_o sin(a \pi x) cos(a \pi y ) sin(a \pi z) \nonumber \\
  w &=& -w_o sin(a \pi x) sin(a \pi y ) cos(a \pi z) \nonumber \\
  p &=& -\frac{p_o}{4}( cos(2 a \pi x) + cos(2 a \pi y ) + cos(2 a \pi z) )  \nonumber \\
  z &=& +z_o cos(a_z \pi x) cos(a_z \pi y ) cos(a_z \pi z)  \nonumber \\
\label{3dNonIso}
\end{eqnarray}
%
The equation of state is simply the standard inverse mixture fraction 
property expression for density, 
\begin{equation}
  \rho = \frac{1} {\frac{z}{rho^P} + \frac{1-z}{rho^S} }
\label{idealGasEOS}
\end{equation}

The simulation is run on a three-dimensional domain ranging from -0.05:+0.05 with
constants $a, a_z, \rho^p, \rho^s, Sc, \mu$ equal to (20, 10, 0.1, 1.0, 0.8, 0.001),
respectively.

At reference conditions, the density is that of the primary condition (0.1). 
The effects of buoyancy are also provided by an arbitrary gravity vector 
of magnitude of approximately ten, $g_i = (-5, 6, 7)^T$. On this domain, the mixture 
fraction ranges from zero to unity. This test case is designed to support 
the helium plume DNS study with primary and secondary density values of helium
and air, respectively.

Edge- and element-based discretization (P=1) demonstrate second order convergence
in the $L_2$ norm for u, v, w and mixture fraction. This test is captured within the 
variableDensityMMS regression test suite.
