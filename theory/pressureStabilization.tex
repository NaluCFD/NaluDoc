A number of papers describing the pressure stabilization approach that Nalu uses are in the open literature,
Domino,~\cite{Domino:2006, Domino:2008, Domino:2014}.  Nalu supports an incremental
fourth order approximate projection scheme with time step 
scaling. By scaling, it is implied that a time scale based on either the physical time step or 
a combined elemental advection and diffusion time scale based on element length along
with advection and diffusional parameters. An alternative to the approaximate projection
concept is to view the method as a variational multiscale (VMS) method wherebye the momentum residual
augments the continuity equation. This allows for a diagonal entry for the pressure degree of freedom.

Here, the fine-scale momentum residual is written in terms of a projected momentum residual 
evaluated at the Gauss point,
\begin{equation}
  \mathbf{R}(u_i) = (\frac{\partial p} {\partial x_j} - G_j p ).
\label{fineScaleMomentum}
\end{equation}
The above equation is derived simply by writing a fine-scale momentum equation at the Gauss-points 
and using the nodal projected residual to reconstruct the individual terms.
Therefore, the continuity equation solved, using the VMS-based projected momentum residual, is

\begin{equation}
\int \frac{\partial \bar{\rho}} {\partial t}\, dV
+ \int \left( \bar{\rho} \hat{u}_i + \tau G_i \bar{P} \right) n_i\, dS
  = \int \tau \frac{\partial \bar{P}}{\partial x_i} n_i\, dS.
\end{equation}
%
Above, $G_i \bar{P}$ is defined as a L2 nodal projection of the pressure gradient. Note that the notion of 
a provisional velocity, $\hat u_i$, is used to signify that this velocity is the product of the momentum 
solve. The difference between the projected nodal gradient interpolated to the gauss point and the 
local gauss point pressure gradient provides a fourth order pressure stabilization
term. This term can also be viewed as an algebraic form for the momentum residual. For the continuity
equation only, a series of element-based options that shift the integration points to the edges of the 
iterated element is an option.

\subsection{The Role of $\dot m$}

In all of the above equations, the advection term is written in terms of a linearized 
mass flow rate including a sum over all subcontrol surface integration points,
Eq.~\ref{advForm}. The mass flow rate includes the full set of stabilization terms
obtained from the continuity solve,
\begin{equation}
\dot m = \left(\bar{\rho} \hat{u}_i + \tau G_i \bar{P} 
  -\tau \frac{\partial \bar{P}}{\partial x_i}\right) n_i\, dS.
\end{equation}
The inclusion of the pressure stabilization terms in the advective
transport for the primitives is a required step for ensuring that
the advection velocity is mass conserving. In practive, the mass flow
rate is stored at each integration point in the mesh (edge midpoints for the
edge-based scheme and subcontrol surfaces for the element-based scheme).
When the mixed CVFEM/EBVC scheme is used, the continuity equation solves for
a subcontrol-surface value of the mass flow rate. These values are assembled to the
edge for use in the EBVC discretization approach. Therefore, the storage for mass
flow rate is higher.

