The Nalu code base is a c++ code-base that significantly leverages
the Sierra Toolkit and Trilinos infrastructure. This section is designed to
provide a high level overview of the underlying abstractions that the 
code base exercises. For more detailed code information, the developer is referred to
the Trilinos project (github.com). In
the sections that follow, only a high level overview is provided. 

The developer might also find useful examples in the NaluUnit github repository
as it contains a number of specialized implementations that are very small in nature.
In fact, the Nalu code base emerged as a small testbed unit test to evaluate the STK
infrastructure. Interestingly, the first 
``algorithm'' implementation was a simple $L_2$ projected nodal gradient. This effort 
involved reading in a mesh, registering a nodal (vector) field, iterating elements 
and exposed surfaces to assemble the projected nodal gradient to the nodes of the mesh
(in parallel). When evaluating kokkos, this algorithm was also used to learn about 
the parallel NGP abstraction provided.

\subsection{Sierra Toolkit Abstractions}
Consider a typical mesh that consists of nodes, sides of elements and elements.
Such a mesh, when using the Exodus standard, will liekly be represented by a 
collection of ``element blocks'', ``sidesets'' and, possibly, ``nodesets''. The 
definition of the mesh (generated by the user through commercial meshing packages such as
pointwise or ICM-CFD) will provide the required spatial definitions of the volume physics 
and the required boundary conditions.

An element block is a homegeneous collection of elements of the same
underlying topology, e.g., HEXAHEDRAL-8. A sideset is a set of exposed element faces
on which a boundary condition is to be applied. Finally, a nodeset is a collection of 
nodes. In general, nodesets are possibly output entities as the code does not exercise
enforcing physics or boundary conditions on nodesets. Although Nalu supports an edge-based
scheme, an edge, which is an entity connecting two nodes, is not part of the Exodus standard
and must be generated within the STK infrastructure. Therefore, a particular discretization
choice may require $stk::mesh::Entity$ types of element, face (or side), edge and node.

Once the mesh is read in, a variety of routine operations are generally required. For example,
a low-Mach physics equation set may want to be applied to $block_1$ while inflow, open, symmetry, 
periodic and wall boundary conditions can be applied to a variety of sidesets. For example, 
$surface_1$ might be of an ``inflow'' type. Therefore, the high level set of requirements on a mesh 
infrastructure might be to allow one to iterate parts of the mesh and, in the end, assemble
a quantity to a nodal or elemental field. 

\subsubsection{Meta and Bulk Data}
Meta and Bulk data are simply STK containers. MetaData is used to extract parts, 
extract ownership status, determine the side rank, field declaration, etc. 
BulkData is used to extract buckets, extract upward and downward connectivities 
and determine node count for a given entity.

\subsubsection{Parallel Rules}
In STK, elements are locally owned by a single rank. Elements may be ghosted to other parallel ranks
through STK custom ghosting. Exposed faces are locally owned by the lower parallel rank. Nodes
are also locally owned by the lower parallel rank and can also be shared by all parallel ranks touching
them. Edges and internal faces (element:face:element connectivity) have the same rule of locally 
owned/shared and can also be ghosted. Again, edges and internal faces must be created by existing
STK methods should the physics algorithm require them. In Nalu, the choice of element-based 
or edge-based is determined within the input file.

\subsubsection{Connectivity}
In an unstructured mesh, connectivity must be built from the mesh and can not be assumed to 
follow an assumed ``i-j-k'' data layout, i.e., structured. In general, one speaks of downward and upward 
relationships between the underlying entities. For example, if one has a particular element, one might
like to extract all of the nodes connected to the element. Likewise, this represents a common opporation
for faces and edges. Such examples are those in which downward relationships are required. 
However, one might also have a node and want to extract all of the connected elements to this 
node (consider some sort of patch recovery algorithm). STK provides the ability to extract 
such connectivities. In general, full downward and upward connectivites are created.

For example, consider an example in which one has a pointer to an element and wants to 
extract the nodes of this element. At this point, the developer has not been exposed to 
abstractions such as buckets, selectors, etc. As such, this is a very high level overview
with more details to come in subsequent sections. Therefore, the scope below is to assume that
from an element-k of a ``bucket'', b[k] (which is a collection of homegeneous RANK-ed entities) 
we will extract the nodes of this element using the STK bulk data.

\begin{lstlisting}

// extract element from this bucket
stk::mesh::Entity elem = b[k];

// extract node relationship from bulk data
stk::mesh::Entity const * node_rels = bulkData_.begin_nodes(elem);
int num_nodes = bulkData_.num_nodes(elem);

// iterate nodes
for ( int ni = 0; ni < num_nodes; ++ni ) {
  stk::mesh::Entity node = node_rels[ni];
  
  // set connected nodes
  connected_nodes[ni] = node;
  
  // gather some data, e.g., density at state Np1, into a local workset pointer
  p_density[ni] = *stk::mesh::field_data(densityNp1, node );
}

\end{lstlisting}

\subsubsection{Parts}
As noted before, a $stk::mesh::Part$ is simply an abstraction that describes a 
set of mesh entities. If one has the name of the part from the mesh data base,
one may extract the part. Once the part is in hand, one may iterate the underlying set of
entities, walk relations, assemble data, etc.

The following example simply extracts a part for each vector of names that lives in
the vector $targetNames$ and provides this part to all of the underlying equations
that have been created for purposes of nodal field registration. Parts of the mesh
that are not included within the $targetNames$ vector would not be included in the 
field registration and, as such, if this mising part was used to extract the data, an 
error would occur.

\begin{lstlisting}

for ( size_t itarget = 0; itarget < targetNames.size(); ++itarget ) {
  stk::mesh::Part *targetPart = metaData_.get_part(targetNames[itarget]);

  // check for a good part
  if ( NULL == targetPart ) {
    throw std::runtime_error("Trouble with part " + targetNames[itarget]);
  }
  else {
    EquationSystemVector::iterator ii;
    for( ii=equationSystemVector_.begin(); ii!=equationSystemVector_.end(); ++ii )
    (*ii)->register_nodal_fields(targetPart);
  }
}

\end{lstlisting}

\subsubsection{Selectors}
In order to arrive at the percise parts of the mesh and entities on which one desires
to operate, one needs to ``select'' what is useful. The STK selector infrastructure provides this.

In the following example, it is desired to obtain a selector that containst all of the parts of 
interest to a physics algorithm that are locally owned and active.

\begin{lstlisting}

// define the selector; locally owned, the parts I have served up and active
stk::mesh::Selector s_locally_owned_union = metaData_.locally_owned_part()
  & stk::mesh::selectUnion(partVec_) 
  & !(realm_.get_inactive_selector());

\end{lstlisting}

\subsubsection{Buckets}
Once a selector is defined (as above) an abstraction to provide access to the type of 
data can be defined. In STK, the mechanism to iterate entities on the mesh is through the 
bucket interface. A bucket is a homegenous collection of $stk::mesh::Entity$.

In the below example, the selector is used to define the bucket of entities that are provided
to the developer.

\begin{lstlisting}
// given the defined selector, extract the buckets of type ``element''
stk::mesh::BucketVector const& elem_buckets 
  = bulkData_.get_buckets( stk::topology::ELEMENT_RANK, 
                           s_locally_owned_union );

// loop over the vector of buckets 
for ( stk::mesh::BucketVector::const_iterator ib = elem_buckets.begin();
      ib != elem_buckets.end() ; ++ib ) {
  stk::mesh::Bucket & b = **ib ;
  const stk::mesh::Bucket::size_type length   = b.size();

  // extract master element (homogeneous over buckets)
  MasterElement *meSCS = realm_.get_surface_master_element(b.topology());
  
  for ( stk::mesh::Bucket::size_type k = 0 ; k < length ; ++k ) {
    
    // extract element from this bucket
    stk::mesh::Entity elem = b[k];
    
    // etc...
  }
}

\end{lstlisting}

The look-and-feel for nodes, edges, face/sides is the same, e.g.,

\begin{description}
\item[$\bullet$ for nodes:]
\end{description}

\begin{lstlisting}
// given the defined selector, extract the buckets of type ``node''
stk::mesh::BucketVector const& node_buckets 
  = bulkData_.get_buckets( stk::topology::NODE_RANK, 
                           s_locally_owned_union );

// loop over the vector of buckets 
\end{lstlisting}

\begin{description}
\item[$\bullet$ for edges:]
\end{description}

\begin{lstlisting}
// given the defined selector, extract the buckets of type ``edge''
stk::mesh::BucketVector const& edge_buckets 
  = bulkData_.get_buckets( stk::topology::EDGE_RANK, 
                           s_locally_owned_union );

// loop over the vector of buckets 
\end{lstlisting}


\begin{description}
\item[$\bullet$ for faces/sides:]
\end{description}

\begin{lstlisting}
// given the defined selector, extract the buckets of type ``face/side''
stk::mesh::BucketVector const& face_buckets 
  = bulkData_.get_buckets( metaData_.side_rank(), 
                           s_locally_owned_union );

// loop over the vector of buckets 
\end{lstlisting}


\subsubsection{Field Data Registration}
Given a part, we would like to declare the field and put the field on the part of interest.
The developer can register fields of type elemental, nodal, face and edge of desired size. 

\begin{description}
\item[$\bullet$ nodal field registration:]
\end{description}

\begin{lstlisting}
void
LowMachEquationSystem::register_nodal_fields(
  stk::mesh::Part *part)
{
  // how many states? BDF2 requires Np1, N and Nm1
  const int numStates = realm_.number_of_states();

  // declare it
  density_ 
    =  &(metaData_.declare_field<ScalarFieldType>(stk::topology::NODE_RANK, 
                                                 "density", numStates));

  // put it on this part
  stk::mesh::put_field(*density_, *part);
}
\end{lstlisting}

\begin{description}
\item[$\bullet$ edge field registration:]
\end{description}

\begin{lstlisting}
void
LowMachEquationSystem::register_edge_fields(
  stk::mesh::Part *part)
{
  const int nDim = metaData_.spatial_dimension();
  edgeAreaVec_ 
    = &(metaData_.declare_field<VectorFieldType>(stk::topology::EDGE_RANK, 
                                                "edge_area_vector"));
  stk::mesh::put_field(*edgeAreaVec_, *part, nDim);
}
\end{lstlisting}

\begin{description}
\item[$\bullet$ side/face field registration:]
\end{description}

\begin{lstlisting}
void
MomentumEquationSystem::register_wall_bc(
  stk::mesh::Part *part,
  const stk::topology &theTopo,
  const WallBoundaryConditionData &wallBCData)
{
  // Dirichlet or wall function bc
  if ( wallFunctionApproach ) {
    stk::topology::rank_t sideRank 
      = static_cast<stk::topology::rank_t>(metaData_.side_rank());
    GenericFieldType *wallFrictionVelocityBip 
      =  &(metaData_.declare_field<GenericFieldType>
          (sideRank, "wall_friction_velocity_bip"));
    stk::mesh::put_field(*wallFrictionVelocityBip, *part, numIp);
  }
}
\end{lstlisting}

\subsubsection{Field Data Access}
Once we have the field registered and put on a part of the mesh, we can extract the field
data anytime that we have the entity in hand. In the example below, we extract nodal field
data and load a workset field.

To obtain a pointer for a field that was put on a node, edge face/side or element field, 
the string name used for declaration is used in addition to the field template type,

\begin{lstlisting}
VectorFieldType *velocityRTM 
  = metaData_.get_field<VectorFieldType>(stk::topology::NODE_RANK, 
                                        "velocity");
ScalarFieldType *density 
  = metaData_.get_field<ScalarFieldType>(stk::topology::NODE_RANK, 
                                        "density");}

VectorFieldType *edgeAreaVec 
  = metaData_.get_field<VectorFieldType>(stk::topology::EDGE_RANK, 
                                        "edge_area_vector");

GenericFieldType  *massFlowRate
  = metaData_.get_field<GenericFieldType>(stk::topology::ELEMENT_RANK, 
                                         "mass_flow_rate_scs");

GenericFieldType *wallFrictionVelocityBip_ 
  = metaData_.get_field<GenericFieldType>(metaData_.side_rank(), 
                                         "wall_friction_velocity_bip");
\end{lstlisting}

\subsubsection{State}
For fields that require state, the field should have been declared with the 
proper number of states (see field declaration section). Once the field pointer is 
in hand, the specific field with state is easily extracted,

\begin{lstlisting}
ScalarFieldType *density 
  = metaData_.get_field<ScalarFieldType>(stk::topology::NODE_RANK, 
                                        "density");
densityNm1_ = &(density->field_of_state(stk::mesh::StateNM1));
densityN_ = &(density->field_of_state(stk::mesh::StateN));
densityNp1_ = &(density->field_of_state(stk::mesh::StateNP1));
\end{lstlisting}

With the field pointer already in hand, obtaining the particular data
is field the field data method.

\begin{description}
\item[$\bullet$ nodal field data access:]
\end{description}

\begin{lstlisting}
// gather some data (density at state Np1) into a local workset pointer
p_density[ni] = *stk::mesh::field_data(densityNp1, node );
\end{lstlisting}

\begin{description}
\item[$\bullet$ edge field data access:] (from an edge bucket loop with the same selector as defined above)
\end{description}

\begin{lstlisting}
stk::mesh::BucketVector const& edge_buckets 
  = bulkData_.get_buckets( stk::topology::EDGE_RANK, s_locally_owned_union );
for ( stk::mesh::BucketVector::const_iterator ib = edge_buckets.begin();
      ib != edge_buckets.end() ; ++ib ) {
  stk::mesh::Bucket & b = **ib ;
  const stk::mesh::Bucket::size_type length   = b.size();

  // pointer to edge area vector and mdot (all of the buckets)
  const double * av = stk::mesh::field_data(*edgeAreaVec_, b);
  const double * mdot = stk::mesh::field_data(*massFlowRate_, b);
  
  for ( stk::mesh::Bucket::size_type k = 0 ; k < length ; ++k ) {
    // copy edge area vector to a pointer
    for ( int j = 0; j < nDim; ++j )
      p_areaVec[j] = av[k*nDim+j];
     
    // save off mass flow rate for this edge
    const double tmdot = mdot[k];
  }
}
\end{lstlisting}

\subsection{High Level Nalu Abstractions}

\subsubsection{Realm}
A realm holds a particlular physics set, e.g., low-Mach fluids. Realms are coupled
loosely through a transfer operation. For example, one might have a turbulent 
fluids realm, a thermal heat conduction realm and a PMR realm. The realm also 
holds a BulkData and MetaData since a realm requires fields and parts to solve
the desired physics set.

\subsubsection{EquationSystem}
An equation system holds the set of PDEs of interest. As Nalu uses a pressure projection
scheme with split PDE systems, the pre-defined systems are, LowMach, MixtureFraction,
Enthalpy, TurbKineticEnergy, etc. New monolithic equation system can be easily created and 
plugged into the set of all equation systems.

In general, the creation of each equation system is of arbitrary order, however, in some cases
fields required for MixtureFraction, e.g., $mass_flow_rate$ might have only been registered
on the low-Mach equaiton system. As such, if MixtureFraction is created before LowMachEOS,
an error might be noted. This can be easily resolved by cleaning the code base such that
each equation system is ``autonomous''.

Each equation system has a set of virtual methods expected to be implemented. These include, however,
are not limited to registration of nodal fields, edge fields, boundary conditions of fixed
type, e.g., wall, inflow, symmetry, etc.

