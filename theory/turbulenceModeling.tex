
Unlike a RANS approach which models most or all of the turbulent
fluctuations, LES directly solves for all resolved turbulent length 
scales and only models the smallest scales below the grid size.  In this
way, a majority of the problem-dependent, energy-containing turbulent
structure is directly solved in a model-free fashion.  The subgrid scales
are closer to being isotropic than the resolved scales, and they generally
act to dissipate turbulent kinetic energy cascaded down from the larger
scales in momentum-driven turbulent flows.  Modeling
of these small scales is generally more straightforward than RANS 
approaches, and overall solutions are usually more tolerant to LES
modeling errors because the subgrid scales comprise such a small portion
of the overall turbulent structure.  While LES is
generally accepted to be much more accurate than RANS approaches for 
complex turbulent flows, it is also significantly more expensive
than equivalent RANS simulations due to the finer grid resolution
required.  Additionally, since LES results in a full unsteady solution,
the simulation must be run for a long time to gather any desired 
time-averaged statistics.  The tradeoff between accuracy and cost 
must be weighed before choosing one method over the other.

The separation of turbulent length scales required for LES is obtained
by using a spatial filter rather than the RANS temporal filter.
This filter has the mathematical form
%
\begin{equation} 
\overline{\phi(\boldsymbol{x},t)} \equiv \int_{-\infty}^{+\infty}
    \phi(\boldsymbol{x}',t) G(\boldsymbol{x}' - \boldsymbol{x})\,
    \mathrm{d}\boldsymbol{x}',
\label{les-filter}
\end{equation}
%
which is a convolution integral over physical space $\boldsymbol{x}$
with the spatially-varying filter function $G$.  The filter function
has the normalization property $\int_{-\infty}^{+\infty}
G(\boldsymbol{x})\, \mathrm{d}\boldsymbol{x} = 1$, and it has a characteristic
length scale $\Delta$ so that it filters out turbulent length scales
smaller than this size.  In the present formulation, a simple ``box filter''
is used for the filter function,
%
\begin{equation}
G(\boldsymbol{x}' - \boldsymbol{x}) = \left\{ \begin{array}{l@{\quad:\quad}l}
    1/V         & (\boldsymbol{x}' - \boldsymbol{x}) \in \mathcal{V} \\
    0           & \mathrm{otherwise} \\
    \end{array} \right.,
\end{equation}
%
where $V$ is the volume of control volume $\mathcal{V}$ whose central 
node is located at $\boldsymbol{x}$.  This is essentially an unweighted
average over the control volume.  The length scale of this filter
is approximated by $\Delta = V^\frac{1}{3}$.  This is typically called 
the grid filter, as it filters out scales smaller than the 
computational grid size.

Similar to the RANS temporal filter, a variable can be represented in 
terms of its filtered and subgrid fluctuating components as
%
\begin{equation}
\phi = \bar{\phi} + \phi'.
\end{equation}
%
For most forms of the filter function $G(\boldsymbol{x})$,
repeated applications of the grid filter to a variable do not yield the
same result.  In other words, $\bar{\bar{\phi}} \ne 
\bar{\phi}$ and therefore $\overline{\phi'} \ne 0$, unlike with the 
RANS temporal averages.

As with the RANS formulation, modeling is much simplified in the
presence of large density variations if a Favre-filtered approach is used.
A Favre-filtered variable $\tilde{\phi}$ is defined as
%
\begin{equation}
\tilde{\phi} \equiv \frac{ \overline{\rho\phi} }{ \bar{\rho} }
\end{equation}
%
and a variable can be decomposed in terms of its Favre-filtered and
subgrid fluctuating component as
%
\begin{equation}
\phi = \tilde{\phi} + \phi''.
\end{equation}
%
Again, note that the useful identities for the Favre-filtered RANS
variables do not apply, so that $\bar{\tilde{\phi}} \ne \tilde{\phi}$
and $\overline{\phi''} \ne 0$.  The Favre-filtered approach is used for
all LES models in Nalu.

\subsection{Standard Smagorinsky LES Model}

The standard Smagorinsky LES closure model approximates the subgrid
turbulent eddy viscosity using a mixing length-type model, where the 
LES grid filter size $\Delta$ provides a natural length scale.  The
subgrid eddy viscosity is modeled simply as (Smagorinsky)
%
\begin{equation}
\mu_t = \rho \left(C_s \Delta \right)^2 | \tilde {S} |,
\label{mut-smag}
\end{equation}
%
The constant coefficient $C_s$ typically varies between $0.1$ and
$0.24$ and should be carefully tuned to match the problem being solved
(Rogallo and Moin,~\cite{Rogallo:1984}).  The default value of $0.17$ is assigned in the code base.

Although this model is desirable due to its simplicity and efficiency,
care should be taken in its application.  It is known to predict subgrid
turbulent eddy viscosity proportional to the shear rate in the flow,
independent of the local turbulence intensity.  Non-zero subgrid turbulent
eddy viscosity is even predicted in completely laminar regions of the
flow, sometimes even preventing a natural transition to turbulence. The model also
does not asymptotically replicate near wall behavior without either dampening or a
dynamic procedure.

\subsection{Wall Adapting Local Eddy-Viscosity, WALE}

The WALE model of Ducros el al.,~\cite{Ducros:1998}, properly captures the asymptotic
behavior for flows that are wall bounded. In this model, the turbulent viscosity is
given by,
%
\begin{equation}
\mu_t = \rho \left(C_w \Delta \right)^2 \frac{\left( S^d_{ij}S^d_{ij}\right)^{3/2}}{\left( S_{ij}S_{ij}\right)^{5/2} + \left( S^d_{ij}S^d_{ij}\right)^{5/4}},
\label{mut-wale}
\end{equation}
%
with the constant $C_w$ of 0.325 and a standard filter, $\Delta$ related to the volume, $V^{\frac{1}{3}}$. 
The rate of strain tensor is defined as,
\begin{equation}
S_{ij} = \frac{1}{2} \left( \frac{\partial u_i}{\partial x_j} + \frac{\partial u_j}{\partial x_i} \right)
\label{wale-sij}
\end{equation}
while $S^d_{ij}$ is,
\begin{equation}
 S^d_{ij} = \frac{1}{2} \left( g^2_{ij} + g^2_{ji}\right).
\label{wale-sij}
\end{equation}

Finally, the velocity gradient squared ters are
\begin{equation}
 g^2_{ij} = \frac{\partial u_i}{\partial x_k} \frac{\partial u_k}{\partial x_j}
\label{wale-sqij}
\end{equation}
and
\begin{equation}
 g^2_{ji} = \frac{\partial u_j}{\partial x_k} \frac{\partial u_k}{\partial x_i}.
\label{wale-gsqji}
\end{equation}

\subsection{One Equation $k^{sgs}$}
See $k^{sgs}$ pde section.

\subsection{SST RANS Model}
As noted, Nalu does support a SST RANS-based model (the reader is referred to the SST equation set description).

\subsection{Wall Models}
Flows are either expected to be fully resolved or, alternatively, under-resolved
where wall functions are used. A classic law of the wall has been implemented in Nalu. 
Wall models to handle adverse pressure gradients are planned. For more information of the form
of wall models, please refer to the boundary condition section of this manual.



