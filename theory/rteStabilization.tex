The RTE is solved using the method of discrete ordinates using the symmetric
Thurgood quadrature set. The discrete ordinates method is one in which
discrete directions of the intensity are solved. The quadrature order, $N$,
defines the number of ordinate directions that are solved in a given 
iteration. In the case of non-scattering media, this results is a set
of decoupled linear partial differentail equations. For the symmetric
Thurgood set, the number of ordinate directions is given by $8N^2$. Values
of N that are required for suitable accuracy starts at $N=2$ with more than
adequate resolution at $N=4$.

For each ordinate direction, a weight is provided, $w_k$ (not to be confused with the
test function $w$). For each intensity ordinate direction,
$I_k$, integrated quantities such as scalar flux and radiative heat flux are computed as,

\begin{equation}
G = \sum I_k w_k
\end{equation}
and,
\begin{equation}
  q_j = \sum I_k s_j w_k.
\end{equation}

The stabilization that is used in the RTE equation can be placed in the
class of residual-based stabilization. In this particular implementation,
the scaled residual of the RTE equation is added. This implementation
has its roots in the classic variational multiscale (VMS). 

In the VMS framework, the degree of freedom is decomposed in terms of its 
resolved and fine scale, $I+I'$. Without specific definition of the test function, 
the weighted residual statement for the RTE within a VMS framework is given by,
\begin{equation}
   \int w \left( s_i {{\partial} \over {\partial x_i}} \left(I\left(s\right) + I'\left(s\right)\right)
   + (\mu_a + \mu_s) \left(I\left(s\right) + I'\left(s\right)\right) - {{\mu_a \sigma T^4} \over {\pi}} -\frac{\mu_s}{4\pi}G \right) {\rm d}V = 0.
\label{rteVMS}
\end{equation}
%
Grouping resolved and fine scale terms results in an equation takes the form of a standard Galerkin 
contribution in addition to the fine structure statement,
\begin{eqnarray}
   \int w \left( s_i {{\partial} \over {\partial x_i}} I\left(s\right)
   + (\mu_a + \mu_s)I - {{\mu_a \sigma T^4} \over {\pi}} -\frac{\mu_s}{4\pi}G \right) {\rm d}V  \nonumber \\ 
   + \int w \left( s_i {{\partial} \over {\partial x_i}} I'\left(s\right)
   + (\mu_a + \mu_s) I' \right) {\rm d}V  = 0.
\label{rteVMS1}
\end{eqnarray}
%
Note that the isotropic source term has not contributed to the VMS framework other than through
the right hand source term.

In general, gradients in the fine scale quantity are to be avoided. Therefore, the first term
in the second line of Eq.~\ref{rteVMS1} is integrated by parts to yield the following
form (note the boundary term, $\int_\Gamma$ that is shown below is frequently dropped)
\begin{eqnarray}
   \int w \left( s_i {{\partial} \over {\partial x_i}} I\left(s\right)
   + (\mu_a +\mu_s)I - {{\mu_a \sigma T^4} \over {\pi}} -\frac{\mu_s}{4\pi}G \right) {\rm d}V  \nonumber \\ 
   - \int I' s_i {{\partial w} \over {\partial x_i}} {\rm d}V 
   + \int_{\Gamma} w s_i I' n_i {\rm d}S + \int w (\mu_a + \mu_s)I' {\rm d}V  = 0.
\label{rteVMS2}
\end{eqnarray}
The following ansatz, which now includes the classic stabilization parameter, $\tau$,
provides closure of the above fine scale equation,
\begin{equation}
   I' = -\tau\left(s_i {{\partial} \over {\partial x_i}} I\left(s\right) + (\mu_a + \mu_s)I\left(s\right) 
   - {{\mu_a \sigma T^4} \over {\pi}} -\frac{\mu_s}{4\pi}G \right) = -\tau R(s)
\label{fineScaleClosure}
\end{equation}
%
 Substituting Eq.~\ref{fineScaleClosure} into Eq.~\ref{rteVMS2} yields,
\begin{eqnarray}
   \int w \left( s_i {{\partial} \over {\partial x_i}} I\left(s\right)
   + (\mu_a +\mu_s)I - {{\mu_a \sigma T^4} \over {\pi}} -\frac{\mu_s}{4\pi}G \right) {\rm d}V  \nonumber \\ 
   + \int \tau s_i {{\partial w} \over {\partial x_i}} R(s) {\rm d}V 
   - \int_\Gamma \tau w R(s) s_i n_i {\rm d}S - \int \tau w (\mu_a + \mu_s)R(s) {\rm d}V  = 0.
\label{rteVMS3}
\end{eqnarray}
In the above equation, the residual of the intensity equation for ordinate $s$
is denoted by $R(s)$. A compact form of the equation is provided by defining a modified
test function, $\tilde w$, (again note retention of the stabilized boundary term)
\begin{eqnarray}
   \int \tilde w \left( s_i {{\partial} \over {\partial x_i}} I\left(s\right)
   + (\mu_a + \mu_s)I - {{\mu_a \sigma T^4} \over {\pi}} -\frac{\mu_s}{4\pi}G \right) {\rm d}V \nonumber \\
 - \beta \int_\Gamma \tau w R(s) s_i n_i {\rm d}S = 0.
\label{rteVMSCompact}
\end{eqnarray}
where $\tilde w$ is simply equal to,
\begin{equation}
   \tilde w  = w + \tau \left( s_j \frac{\partial w }{\partial x_j} + \alpha (\mu_a + \mu_s)w \right).
\label{modW}
\end{equation}
When $\alpha = -1$, we have the above VMS derivation; for $\alpha = 1$, Galerkin Least Squares is realized;
finally for $\alpha = 0$, we have SUPG. For any formulation other than VMS, the residual contribution
at the boundaries of the domain is dropped ($\beta = 0$).

The full residual-based equation is placed in divergence form,
\begin{eqnarray}
   \int \tilde w \left( {{\partial} \over {\partial x_i}} s_i I\left(s\right)
   + (\mu_a + \mu_s) I\left(s\right) - {{\mu_a \sigma T^4} \over {\pi}} -\frac{\mu_s}{4\pi}G \right) {\rm d}V \nonumber \\
   - \beta \int_\Gamma \tau w R(s) s_i n_i {\rm d}S = 0.
\label{rteDivForm}
\end{eqnarray}
and split into its Galerkin and stabilized contributions,
\begin{eqnarray}
   \int w \left( {{\partial} \over {\partial x_i}} s_i I\left(s\right)
   + (\mu_a + \mu_s )I\left(s\right) - {{\mu_a \sigma T^4} \over {\pi}} -\frac{\mu_s}{4\pi}G \right) {\rm d}V \nonumber \\
 +\int \tau s_j \frac{\partial w }{\partial x_j} R(s) {\rm d}V \nonumber \\
+\alpha\int \tau w (\mu_a + \mu_s)R(s) {\rm d}V \nonumber \\
- \beta \int_\Gamma \tau w R(s) s_i n_i {\rm d}S = 0.
\label{rteDivFormSub}
\end{eqnarray}
Note that the first term in the above equation is integrated by parts,
\begin{equation}
\int w {{\partial} \over {\partial x_i}} s_i I\left(s\right) {\rm d}V = -\int I\left(s\right) s_i \frac{\partial w}{\partial x_i} {\rm d}V 
+ \int_{\Gamma} w s_i I\left(s\right) n_i {\rm d}S.
\end{equation}
Again, the usage of $\Gamma$ provides emphasis that the contribution is a boundary (exposed face) condition. 
Therefore, the full VMS-based stabilized RTE equation is as follows,
\begin{eqnarray}
   \int \left( -I\left(s\right) s_i \frac{\partial w}{\partial x_i} + w [(\mu_a + \mu_s) I\left(s\right) 
- {{\mu_a \sigma T^4} \over {\pi}} -\frac{\mu_s}{4\pi}G ] \right) {\rm d}V  \nonumber \\
+ \int_\Gamma w s_i I\left(s\right) n_i {\rm d}S \nonumber \\
 +\int \tau s_j \frac{\partial w }{\partial x_j} R(s) {\rm d}V \nonumber \\
+\alpha\int \tau w (\mu_a + \mu_s)R(s) {\rm d}V \nonumber \\
- \beta \int_\Gamma \tau w R(s) s_i n_i {\rm d}S = 0.
\label{rteDivFormSub}
\end{eqnarray}

\subsubsection{Definition of the CVFEM test function}
Following the work of Martinez,~\cite{Martinez:2005}, the test function is chosen to be a 
piecewise-constant value within the control volume, $w = w_I$ and zero outside of this control volume (Heaviside).
A key property of this function, as pointed out by Martinez, is that its
gradient is a distribution of delta functions on the control volume
boundary:
\begin{equation}
\frac{\partial w_I}{\partial x_i} = - {\bf n}_I \delta({\bf x} - {\bf x}_{\Gamma_I})
\label{eqn:GradPiecewiseConstant}
\end{equation}
where $\Gamma_I$ is boundary of control volume $I$ and ${\bf n}_I$
is the outward normal on that boundary.  Substituting this relationship into the residual equation provides
the final form of vertex-centered finite volume RTE stabilized equation,

\begin{eqnarray}
   \int I\left(s\right) s_i n_i {\rm d}S + \int \left( (\mu_a + \mu_s ) I\left(s\right) 
   - {{\mu_a \sigma T^4} \over {\pi}} -\frac{\mu_s}{4\pi}G \right) {\rm d}V  \nonumber \\
+ \int_\Gamma s_i I\left(s\right) n_i {\rm d}S \nonumber \\
 - \int \tau R(s) s_i n_i {\rm d}S +\alpha \int \tau (\mu_a +\mu_s) R(s) {\rm d}V -\beta \int_\Gamma \tau R(s) s_i n_i {\rm d}S= 0.
\label{rteSUCVForm}
\end{eqnarray}
Given this equation, either an edge-based or element-based scheme can be used. For $\alpha = 0$ and $\beta = 0$, it 
is noted that classic SUCV is obtained. The second line of Eq.~\ref{rteSUCVForm} represents a boundary contribution. 
This is where the intensity boundary condition (Eq.~\ref{intBc}) is applied. As noted in the RTE equation section, 
when $s_j n_j$ is greater than zero, the interpolated intensity based on
the surface nodal values is used. However, when $s_j n_j$ is less than zero, the intensity boundary condition 
value is used. Since the original RTE equation was integrated 
by parts, a natural surface flux contribution is applied. In alternative discretization approaches, e.g., the
SUPG FEM-based Sierra Thermal Radiation Module: Syrinx code, the RTE is not integrated by parts. Therefore, 
no boundary term exists, and, therefore, a dirichlet bc is applied. At corner nodes, this approach can lead 
to non-intuitive approaches since the corner node might have surface facets that are both incoming and outgoing. 
Weak integration of the flux term eliminated this complexity.

\subsubsection{The form of $\tau$}
The value of the stabilization parameter $\tau$ can take on a variety of forms. 
A classic derivation provides the form of $\tau$ to be broken out into two forms, 
$\tau_{adv} = \frac{h}{2}$ and $\tau_{diff} = \frac{1}{(\mu_a+\mu_s)}$. An ad-hoc blending 
is given by,
\begin{equation}
   \tau  = \left( \frac{1} {\frac{2}{h}^2 + (\mu_a+\mu_s)^2} \right)^ \frac{1}{2}
\label{blendedTau}
\end{equation}

Finally, the classic GFEM form of $\tau$ is given by use of the metric tensor for the element mapping is noted,

\begin{equation}
   \tau = \beta^* [s_i g_{ij}s_j]^{-\frac{1}{2}},
\label{farzin}
\end{equation}
with $\beta^*$ equal to unity for SUCV and $\frac{2}{15^{\frac{1}{2}}}$ for FEM.

\subsection{Pure Edge-based Upwind Method}
The residual-based stabilization apporach can lead to predicting negative intensities. This is simply due 
to the fact that the stabilization approach (SUPG) is a linear approach. Extensions of this residual-based
stabilization to include a discontinuity capturing operator (DCO) are underway. This adds a non-linear stabilization
approach that will, hopefully, eliminate negative intensity predictions. 

Alternatively, a first order upwind approach has been implemented by using EBVC discretization. At this point, 
no higher order upwind extensions have been implemented. For the upwind implementation, the equation solved is,

\begin{eqnarray}
   \int I\left(s\right) s_i n_i {\rm d}S + \int \left( (\mu_a + \mu_s ) I\left(s\right) 
   - {{\mu_a \sigma T^4} \over {\pi}} -\frac{\mu_s}{4\pi}G \right) {\rm d}V  \nonumber \\
+ \int_\Gamma s_i I\left(s\right) n_i {\rm d}S  = 0.
\label{rteSUCVForm}
\end{eqnarray}
In the above equation, the ``advection operator'', $I\left(s\right) s_i n_i {\rm d}S$ is
approximated as using the ``upwind'' intensity, e.g., if $s_j n_j$ is greater than zero, 
the left nodal value is used.

\subsection{Finite Element SUPG Form}
For the FEM, the test function is the standard weighting. Assuming a pure SUPG formulation, i.e.,
$\alpha = \beta = 0$ in Equation~\ref{rteDivFormSub}, thereby reducing the final form to the following:

\begin{eqnarray}
   \int \left( -I\left(s\right) s_i \frac{\partial w}{\partial x_i} + w[(\mu_a + \mu_s) I\left(s\right) 
- {{\mu_a \sigma T^4} \over {\pi}} -\frac{\mu_s}{4\pi}G ] \right) {\rm d}V  \nonumber \\
+ \int_\Gamma w s_i I\left(s\right) n_i {\rm d}S \nonumber \\
 +\int \tau s_j \frac{\partial w }{\partial x_j} R(s) {\rm d}V \nonumber \\
\label{rteFemSUPG}
\end{eqnarray}
The weak boundary condition is applied in a similar manner as with the CVFEM and EBVC form, however,
using the appropriate FEM test test function definition. Finally, the form of $\tau$ follows the above
CVFEM form.
