\subsection{Inflow Boundary Condition}

\subsubsection{Continuity}
Continuity uses a flux boundary condition with the incoming
mass flow rate based on the user specified values for velocity,
\begin{equation}
  \dot{m}_c = \rho^{spec} u^{spec}_j A_j.
\end{equation}
As this is a vertex-based code, at inflow and Dirichlet wall boundary locations,
the continuity equation uses the specified velocity within the inflow boundary condition
block.

\subsubsection{Momentum, Mixture Fraction, Enthalpy, Species, $k_{sgs}$, k and $\omega$}
These degree-of-freedoms (DOFs) each use a Dirichlet value with the specified user value.
For all Dirichlet values, the row is zeroed with a unity placed
on the diagonal. The residual is zeroed and set to the difference
between the current value and user specified value.

\subsection{Wall Boundary Conditions}

\subsubsection{Continuity}
Continuity uses a no-op.

\subsubsection{Momentum}
When resolving the boundary layer, Momentum again uses a no-slip 
Dirichlet condition., e.g., $u_i = 0$. 

In the case of a wall model, a classic wall function is applied.
The wall shear stress enters the discretization of the momentum equations 
by the term
%
\begin{equation}
        \int \tau_{ij} n_j dS = -{F_w}_i .
 \label{wall_shear1}
\end{equation}
%
Wall functions are used to prescribe the value of the wall shear 
stress rather than resolving the boundary layer within the near-wall 
domain. The fundamental momentum law of the wall formulation, assuming 
fully-developed turbulent flow near a no-slip wall, can be written as,
%
\begin{equation}
u^+ = {u_{\|} \over u_{\tau}} 
    = { 1 \over \kappa } \ln \left(Ey^+\right) ,
\label{law_wall}
\end{equation}
%
where $u^+$ is defined by the the near-wall parallel velocity, $u_{\|}$, normalized 
by the wall friction velocity, $u_{\tau}$. The wall friction velocity is
related to the turbulent kinetic energy by,
%
\begin{equation} 
        u_{\tau} = C_\mu^{1/4} k^{1/2}.
\label{utau}
\end{equation}
%
by assuming that the production and dissipation of turbulence is in local 
equilibrium.  The wall friction velocity is also computed given the density and wall shear stress,
\begin{equation} 
        u_\tau = (\frac{\tau_w} {\rho})^{0.5}.
\end{equation}


The normalized perpendicular distance from the point in question to the wall, $y^+$, is defined as
the following:
%
\begin{equation} 
        y^+ = {{ \rho Y_p} \over {\mu }}\left(\tau_w \over \rho \right)^{1/2} 
            = {{ \rho Y_p u_{\tau}} \over {\mu }}.
\label{yplus}
\end{equation}
%
The classical law of the wall is as follows:
%
\begin{equation}
       u^+ = \frac{1}{\kappa} \ln(y^+) + C,
\label{lawOfWall1}
\end{equation}
%
where $\kappa$ is the von Karman constant and $C$ is the dimensionless 
integration constant that varies based on authorship and surface roughness. 
The above expression can be re-written as,
%
\begin{equation}
       u^+ = \frac{1}{\kappa} \ln(y^+) + \frac{1}{\kappa} \ln(\exp(\kappa C)),
\label{lawOfWall2}
\end{equation}
%
or simplified to the following expression:
%
\begin{align}
       u^+ &= \frac{1}{\kappa} \left(\ln(y^+) + \ln(\exp(\kappa C))\right) \\
           &= \frac{1}{\kappa} \ln(E y^+).
\label{lawOfWall3}
\end{align}
%
In the above equation, $E$, is referred to in the text as the dimensionless wall 
roughness parameter and is described by,
%
\begin{equation}
       E = \exp(\kappa C).
\label{ElogParam}
\end{equation}
%
In Nalu, $\kappa$ is set to the value of 0.42 while the 
value of $E$ is set to 9.8 for smooth walls\footnote{White 
suggests values of $\kappa=0.41$ and $E=7.768$.}. 
The viscous sublayer is assumed to extend to a value of $y^+$ = 11.63.

The wall shear stress, $\tau_w$, can be expressed as,
%
\begin{equation} 
        \tau_w = \rho u_\tau^2 = \rho u_\tau {{u_\|} \over {u^+}}
               = { {\rho \kappa u_{\tau}}  \over {\ln \left(Ey^+\right) } }u_\|
               = \lambda_w u_\| ,
\label{wall_shear_trb}
\end{equation}
%
where $\lambda_w$ is simply the grouping of the factors from the law of the wall.
For values of $y^+$ less than 11.63, the wall shear stress is given by,
%
\begin{equation}
        \tau_w =  \mu {u_\| \over Y_p} .
\label{wall_shear_lam}
\end{equation}
%
The force imparted by the wall, for the $i_{th}$ component of velocity,
can be written as,
%
\begin{equation}
        F_{w,i}= -\lambda_w A_w u_{i\|} ,
\label{wall_force_1}
\end{equation}
%
where $A_w$ is the total area over which the shear stress acts.

The use of a general, non-orthogonal mesh adds a slight complexity to 
specifying the force imparted on the fluid by the wall.
As shown in Equation~\ref{wall_force_1}, the velocity component parallel to 
the wall must be determined. Use of the unit normal vector, $n_j$, 
provides an easy way to determine the parallel velocity component by the 
following standard vector projection:
%
\begin{equation}
        \Pi_{ij} = \left [ \delta_{ij} - n_i n_j \right].
\label{proj_oper}
\end{equation}
%
Carrying out the projection of a general velocity,
which is not necessarily parallel to the wall, yields the velocity vector 
parallel to the wall,
%
\begin{equation}
        u_{i\|} = \Pi_{ij} {u}_j = u_i\left(1-{n_i}^2\right)
        -\sum_{j=1;j\neq j}^{n} u_j n_i n_j.
\label{proj_operU}
\end{equation}
%
Note that the component that acts on the particular $i^{th}$ component of 
velocity,
%
\begin{equation}
        -\lambda_w A_w \left(1-n_i n_i\right) u_{i\|} ,
\label{implicit_shear}
\end{equation}
%
provides a form that can be potentially treated implicitly; i.e., in a way to 
augment the diagonal dominance of the central coefficient of the $i^{th}$ 
component of velocity. The use of residual form adds a slight complexity to
this implicit formulation only in that appropriate right-hand-side source terms
must be added.

\subsubsection{Mixture Fraction}
If a value is specified for each quantity within the wall boundary condition block, a Dirichlet condition
is applied. If no values are specified, a zero flux condition is applied.

\subsubsection{Enthalpy}
If the temperature is specified within the wall boundary condition block, a Dirichlet condition
is always specified. Wall functions for enthalpy transport have not yet been implemented. 

The simulation tool supports multi-physics coupling via conjugate heat transfer and radiative heat transfer.
Coupling parameters required for the thermal boundary condition are post processed by the fluids or PMR Realm.
For conjugate and radiative coupling, the thermal solve provides the surface temperature. From the surface temperature, 
a wall enthalpy is computed and used.

\subsubsection{Thermal Heat Conduction}
If a temperature is specified in the wall block, and the surface is not an interface condition, then a Dirichlet
approach is used. If conjugate heat transfer is included, then
the boundary condition applied is as follows,

\begin{equation}
     -\kappa \frac{\partial T} {\partial x_j} n_j dS = h(T-T^o)dS,
\end{equation}
where $h$ is the heat transfer coefficient and $T^o$ is the reference
temperature. The details of how these quantities are computed are currently omitted
in this manual. In general, the quantities are post processed from the fluids temperature field. A surface-based
gradient is computed on the boundary face. Nodes on the face augment a heat transfer coefficient field while
nodes off the face contribute to a reference temperature. 

For radiative heat transfer, the boundary condition 
applied is as follows:
\begin{equation}
     -\kappa \frac{\partial T} {\partial x_j} n_j dS = \epsilon (\sigma T^4 - H) dS,
\end{equation}
where $H$ is again the irradiation provided by the RTE solve.

If no temperature is specified or an adiabatic line command is used, a zero flux condition is applied.

\subsubsection{Species}
If a value is specified for each quantity within the wall boundary condition block, a Dirichlet condition
is applied. If no values are specified, a zero flux condition is applied.

\subsection{Atmospheric Boundary Layer Surface Conditions}
\subsubsection{Monin-Obukhov Theory}
Consider atmospheric flow over a flat but non-smooth surface; the
coordinate system convention is that flow is along the $x$-axis, while
the $z$-axis is oriented normal to the surface.  The surface layer is
the relatively thin layer near the surface where strong wind and
temperature gradients exist.  Turbulence within this layer can be
generated through mechanisms of both shear and thermal convection; the
relative contributions of these two mechanisms is determined by the
stability state of the atmosphere.  The stability state is
characterized by the Monin-Obukhov length:
\begin{equation}
 L = - \frac{u_\tau^3 \theta_{ref}}{\kappa g (\overline{w^\prime
     \theta^\prime})_s};
\end{equation}
$u_\tau$ is the friction velocity, defined as the
square root of the magnitude of the Reynolds shear stress at
the surface, or
\begin{equation}
 u_\tau = \left( \overline{w^\prime u^\prime}^2 + \overline{w^\prime
   u^\prime}^2 \right)^{1/4} = \sqrt{\frac{\tau_s}{\rho_s}}
\end{equation}
$\theta_{ref}$ is a reference (virtual potential) temperature associated with the air
within the surface layer; for example, the average temperature within
the surface layer.  $\kappa \approx 0.41$ is the von Karman constant,
and $g$ is the acceleration of gravity.  $\overline{w^\prime
     \theta^\prime}_s$ is the surface turbulent temperature flux.  Both the
turbulent shear stress and turbulent temperature flux are approximately
constant within the surface layer.

Applying a gradient diffusion model for the turbulent temperature flux leads to:
\begin{equation}
 \overline{w^\prime \theta^\prime}_s = -k_T \pd{\theta}{z}
\end{equation}.
The sign of $L$ is then connected to the sign of the temperature
gradient within the surface layer.  Three regimes are delineated:
\begin{itemize}
  \item $\frac{1}{L} > 0, \quad \pd{\theta}{z} > 0$~~~stable
    stratification
  \item $\frac{1}{L} = 0, \quad \pd{\theta}{z} = 0$~~~neutral
    stratification
  \item $\frac{1}{L} < 0, \quad \pd{\theta}{z} < 0$~~~unstable stratification
\end{itemize}

Monin-Obukhov theory postulates the following similarity laws for mean
velocity parallel to the surface and temperature,
\begin{equation} \label{dudz}
\frac{\kappa z}{u_\tau}\pd{\overline{u}_{||}}{z} =
\phi_m\left(\frac{z}{L}\right),
\end{equation}
\begin{equation} \label{dTdz}
\frac{\kappa z u_\tau}{\overline{w^\prime \theta^\prime}_s}
\pd{\overline{\theta}}{z} = \phi_h\left(\frac{z}{L}\right),
\end{equation}
where the forms of the non-dimensional functions $\phi_m$ and $\phi_h$ are determined
from empirical observations. Analytical functions have been fit to the
data; these are not given here, rather, we present the integrated form
of (\ref{dudz}) and (\ref{dTdz}), since these are the forms required
by the code implementation.

For neutral stratification, $\phi_m = 1$ and we recover the
logarithmic profile for a ``fully rough'' surface,
\begin{equation} \label{vel_neutral}
\overline{u}_{||}(z) = \frac{u_\tau}{\kappa}\ln\frac{z}{z_0},
\end{equation}
where $z_0$ is the characteristic roughness height.  Note that viscous
scaling involving surface viscosity and density properties is not
required with this form of the logarithmic profile, since the
roughness height is large enough to eliminate the presence of a
laminar sublayer and buffer layer.

For stable stratification, the surface layer profiles take the form
\begin{equation} \label{vel_stable}
 \overline{u}_{||}(z) = \frac{u_\tau}{\kappa}\left(\ln\frac{z}{z_0} +
 \gamma_m\frac{z}{L}\right)
\end{equation}
\begin{equation}  \label{temp_stable}
\overline{\theta}(z) = \overline{\theta}(z_0) +
\frac{\theta_*}{\kappa} \left(\alpha_h\ln\frac{z}{z_0} +
\gamma_h\frac{z}{L}\right)
\end{equation}
$\theta_*$ is calculated from the temperature flux and friction velocity as
$\theta_* = -\frac{\overline{w^\prime \theta^\prime}_s}{u_\tau}$, and
$\gamma_m$, $\alpha_h$, and $\gamma_h$ are constants specified below.

For unstable stratification, the surface layer profiles take the form
\begin{equation} \label{vel_unstable}
  \overline{u}_{||}(z) = \frac{u_\tau}{\kappa}\left(\ln\frac{z}{z_0} -
  \psi_m\left(\frac{z}{L}\right)\right)
\end{equation}
\begin{equation} \label{temp_unstable}
  \overline{\theta}(z) = \overline{\theta}(z_0) +
  \alpha_h\frac{\theta_*}{\kappa}\left(\ln\frac{z}{z_0} -
  \psi_h\left(\frac{z}{L}\right)\right)
\end{equation}
where
\begin{equation} \label{psi_m}
 \psi_m\left(\frac{z}{L}\right) = 2\ln\frac{1 + x}{2} + \ln\frac{1 + x^2}{2} - 2\tan^{-1}x +
 \frac{\pi}{2}, \quad x = \left(1 - \beta_m\frac{z}{L}\right)^{1/4},
\end{equation}
\begin{equation} \label{psi_h}
 \psi_h\left(\frac{z}{L}\right) = \ln\frac{1 + y}{2}, \quad y = \left(1 -
 \beta_h\frac{z}{L}\right)^{1/2}.
\end{equation}

The constants used in (\ref{vel_stable}) -- (\ref{psi_h}) are \cite{Dyer:74}
\begin{equation}
  \kappa = 0.41,~~\alpha_h =
  1,~~\beta_m=16,~~\beta_h=16,~~\gamma_m=5.0,~~\gamma_h=5.0.
\end{equation}

\subsubsection{ABL Wall Function}
The equations from the preceeding section can be used to formulate a
wall function boundary condition for simulation of atmospheric
boundary layers.  The user-specified inputs to this boundary condition
are: roughness length, $z_0$, and surface heat flux, $q_s =
\rho C_p \overline{w^\prime \theta^\prime})_s$.  The surface layer profile
model is evaluated for each surface boundary flux integration point;
the wall-normal distance of the ``first point off the wall'' is taken
to be one fourth of the length of the nearest edge intersecting the
boundary face.  The boundary condition is specified weakly through the
imposition of a surface shear stress and surface heat flux.

The procedure for applying the boundary condition is as follows.
\begin{enumerate}
\item Determine the stratification state of the boundary layer by
  calculating the sign of the Monin-Obukhov length scale.
\item Solve the appropriate profile equation, either
  (\ref{vel_neutral}), (\ref{vel_stable}), or (\ref{vel_unstable}),
  for the friction velocity $u_\tau$.  For the neutral case, $u_\tau$ can be
  solved for directly.  For the stable and unstable cases, $u_\tau$ must
  be solved for iteratively because $L$ appears in these equations and
  $L$ depends on $u_\tau$.
\item The surface shear stress is calculated as $\tau_s = \rho_s u_\tau^2$.
  For calculating left-hand-side Jacobian entries, the following form
  for the surface shear stress is used:
  \begin{equation} \label{tau_s_useful}
   \tau_{s_i} = \lambda_s u_{||_i} = \frac{\kappa\rho
     u_\tau}{\log(z/z_0) - \psi^\prime (z/L)},
  \end{equation}
  where $\psi^\prime$ is zero for a neutral profile, $-\gamma_m z/L$
  for a stable profile, and $\psi_h(z/L)$ for an unstable profile.
  The Jacobian entries follow directly from this form.
\item The user specified surface heat flux is applied to the enthalpy
  equation.  Evaluation of surface temperature is not required for
  the boundary condition specification.  However, if surface
  temperature is required for evaluation of other quantities within
  the code, the appropriate surface layer temperature profile should
  be used, either (\ref{temp_stable}) or (\ref{temp_unstable}).
\end{enumerate}

\subsubsection{Moeng Wall Function}
The Monin-Obukhov expressions only truly hold in a mean sense, and are not
necessarily valid when used to specify an instantaneous value for the
surface shear stress in a large eddy simulation. Moeng \cite{Moeng:84}
developed a local surface stress condition that utilizes
horizontally-averaged quantities, for which the M-O relationships are
assumed to hold.  This boundary condition is derived by first assuming
that the local tangential shear stress vector can be written using a
drag law:
\begin{equation} \label{draglaw}
  \mathbf{\tau}_s  = C_D u_{{||}_p} \mathbf{u}_{{||}_p}
\end{equation}
Here, $C_D$ is the drag coefficient, $\mathbf{u}_{{||}_p}$ is the
surface-tangential velocity vector evaluated at the near-surface
discretization point, and $u_{{||}_p}$ denotes the magnitude of this
velocity vector.

An expression for $\tau_s$ is derived in the Appendix of Moeng \cite{Moeng:84}, by
writing the velocity as the sum of a horizontally averaged mean
component and a fluctuation about this mean:
\begin{equation}
  \mathbf{u}_{{||}_p} = \left< \mathbf{u}_{{||}_p} \right> +
  \mathbf{u}_{{||}_p}^{\prime\prime}
\end{equation}
There are two main assumptions in the derivation.  The first is that
the instantaneous version of the drag law~(\ref{draglaw}) is
identical to the horizontally-averaged version.  The second assumption
is\footnote{Or, at least, that the difference between these quantities
  is small relative to other terms, see Moeng \cite{Moeng:84}.
}
\begin{equation}
  u_{{||}_p}^{\prime\prime} \mathbf{u}_{{||}_p}^{\prime\prime} \approx
  \left< u_{{||}_p}^{\prime\prime} \mathbf{u}_{{||}_p}^{\prime\prime}
  \right>
\end{equation}

After algebraic manipulations, the resulting vector expression is
\begin{equation} \label{moeng_bc}
  \mathbf{\tau}_s = \left< \mathbf{\tau}_s \right> \left( \frac{u_{{||}_p}
  \left< \mathbf{u}_{{||}_p} \right> +  \left< u_{{||}_p} \right>
  \left[ \mathbf{u}_{{||}_p} - \left< \mathbf{u}_{{||}_p} \right>
    \right]}{\left<  u_{{||}_p} \right> \left< \mathbf{u}_{{||}_p}
    \right>}\right)
\end{equation}

The procedure to calculate the surface stress at a boundary
integration point is as follows.

\begin{enumerate}
 \item Calculate the horizontally-averaged quantities
   $\left<u_{{||}_p}\right>$ and $\left<\mathbf{u}_{{||}_p}\right>$.
 \item Use the M-O velocity profile relationships \cite{Dyer:74} to
   calculate an average friction velocity $u_\tau$.
 \item Use the relationship (\ref{tau_s_useful}) to calculate the components of  $\left<
   \mathbf{\tau}_s \right>$.
 \item Calculate the local surface shear stress using~(\ref{moeng_bc}).
\end{enumerate}

Jacobian entries are required to populate the left-hand side matrix
for the terms resulting from (\ref{moeng_bc}).  For convenience, we
write the vector quantities as tensors with subscripts denoting the
vector component indices.  We need an expression for the sensitivity
of the shear stress, applied at the boundary face integration point,
to the velocity components at the $l^{th}$ grid node, or
$\dd{\tau_{s_i}^{(ip)}}{u_j^(l)}$.

Differentiating (\ref{moeng_bc}) with respect to $\unod{j}$ gives
\begin{equation} \label{sens1}
  \dd{\tau_{s_i}^{(ip)}}{\unod{j}} =
  \frac{\left<\tau_s^{(ip)}\right>_i}{\left<\utan^{(ip)}\right>}
  \dd{\utan^{(ip)}}{\unod{j}} +
  \frac{\left<\tau_s^{(ip)}\right>_i}{\left<{\utan}_i^{(ip)}\right>}\dd{{\utan}_i^{(ip)}}{\unod{j}}
\end{equation}

The first term involves a partial derivative of the tangential
velocity magnitude, while the second term involves the partial
derivative of the tangential velocity component.  Applying the chain
rule to the first term gives
\begin{equation} \label{sens2}
 \dd{\utan^{(ip)}}{\unod{j}} = \frac{1}{\utan^{(ip)}}
   {\utan}_k^{(ip)} \dd{{\utan}_k^{(ip)}}{\unod{j}},
\end{equation}
where summation is implied over the repeated index $k$.  The second
partial derivative in (\ref{sens1}) is seen to appear also in
(\ref{sens2}).  It remains to write an expression for this
derivative, which is done by first writing the tangential velocity
vector at the boundary face integration point in terms of the
Cartesian velocity components:
\begin{equation} \label{utan}
{\utan}_i^{(ip)} = (1 - n_i n_j)\delta_{ij} u_i^{(ip)} - n_i n_j (1 -
\delta_{ij}) u_j^{(ip)}
\end{equation}
with summation over the $j$ index.  The integration point velocity
components are calculated from the face nodes using
\begin{equation} \label{ipvel}
 u_i^{(ip)} = \sum_{l=1}^{N_n} \phi^{(l)}(x_{ip})u_i^{(l)}
\end{equation}
Substituting (\ref{ipvel}) into (\ref{utan}), followed by (\ref{utan})
into ({\ref{sens2}) gives
\begin{equation}
 \dd{\utan^{(ip)}}{\unod{j}} = \frac{1}{\utan^{(ip)}}
   {\utan}_k^{(ip)} \sum_{j=1}^3 \sum_{l=1}^{N_n} (1 - n_i n_j)
   \delta_{ij} \phi^{(l)}(x_{ip}) - n_i n_j (1 -
   \delta_{ij})\phi^{(l)}(x_{ip})
\end{equation}

\subsection{Turbulent Kinetic Energy, $k_{sgs}$ LES model}
When the boundary layer is assumed to be resolved, the natural boundary condition is a Dirichlet value of zero, 
$k_{sgs} = 0$. 

When the wall model is used, a standard wall function approach is used with the assumption of equal production and
dissipation.

The turbulent kinetic energy production term is consistent with the law of the wall formulation and can be expressed as,
%
\begin{equation}
        {P_k}_w = \tau_w {{\partial u_{\|}} \over {\partial y}}.
\label{wall_pk_1}
\end{equation}
%
The parallel velocity, $u_{\|}$, can be related to the wall shear stress by,
%
\begin{equation}
        \tau_w {u^+ \over y^+ } = \mu {u_{\|} \over Y_p }.
\label{tauwall_uplus}
\end{equation}
%
Taking the derivative of both sides of Equation~\ref{tauwall_uplus}, and
substituting this relationship into Equation~\ref{wall_pk_1} yields,
%
\begin{equation} 
        {P_k}_w = {\tau_w^2 \over \mu} {{\partial u^+} \over {\partial y^+}}.
\label{wall_pk_2}
\end{equation}
%
Applying the derivative of the law of the wall formulation, Equation~\ref{law_wall},
provides the functional form of ${\partial u^+ / \partial y^+}$,
%
\begin{equation} 
        {\partial u^+ \over \partial y^+}
      = {\partial \over \partial y^+}
       \left[{ 1 \over \kappa } \ln \left(Ey^+\right) \right]
      = {1 \over \kappa y^+}.
\label{dlaw_wall}
\end{equation}
%
Substituting Equation~\ref{dlaw_wall} within Equation~\ref{wall_pk_2} yields
a commonly used form of the near wall production term,
%
\begin{equation}
        {P_k}_w = {{\tau_w}^2 \over \rho\kappa u_{\tau} Y_p}.
\label{wall_pk_3}
\end{equation}
%
Assuming local equilibrium, $P_k = \rho\epsilon$, and using
Equation~\ref{wall_pk_3} and Equation~\ref{utau} provides the form 
of wall shear stress is given by,
%
\begin{equation}
        \tau_w = \rho C_\mu^{1/2} k.
\label{wall_tauw_equil}
\end{equation}
%
Under the above assumptions, the near wall value for turbulent kinetic 
energy, in the absence of convection, diffusion, or accumulation is given by,
%
\begin{equation}
   k = {{u_\tau^2} \over {C_\mu^{1/2}}}.
\label{wall_tke}
\end{equation}

This expression for turbulent kinetic energy is evaluated at the boundary faces of the
exposed wall boundaries and is area-assembled to the nodal value for use in a Dirichlet condition.

\subsubsection{Turbulent Kinetic Energy and Specific Dissipation SST Low Reynolds Number Boundary conditions}

For the turbulent kinetic energy equation, the wall boundary conditions follow that described for 
the $k_{sgs}$ model, i.e., $k=0$.

A Dirichlet condition is also used on $\omega$.  For this boundary condition, the $\omega$ equation depends only on the near-wall 
grid spacing.  The boundary condition is given by,
\begin{equation}
\omega = {6 \nu \over \beta_1 y^{2}},
\end{equation}
which is valid for $y^{+} < 3$. 

\subsubsection{Turbulent Kinetic Energy and Specific Dissipation SST High Reynolds Number Boundary conditions}
The high Reynolds approach uses the law of the wall assumption and also follows the description 
provided in the wall modeling section with only a slight modification in constant syntax,

\begin{equation}
k = {u_{\tau}^{2} \over \sqrt{\beta^*}}.
\label{wallModelTke}
\end{equation}
% 
In the case of $\omega$, an analytic expression is known in the log layer:
\begin{equation}
\omega = {u_{\tau} \over \sqrt{\beta^*} \kappa y},
\end{equation}
which is independent of $k$.  Because all these expressions require $y$ to be in the log layer, 
they should absolutely not be used unless it can be guaranteed that $y^{+} > 10$, and $y^{+} > 25$ is preferable.
%
Automatic blending is not currently supported.

\subsubsection{Solid Stress}
The boundary conditions applied are either force provided by a static pressure, 

\begin{equation}
 F^n_i = \int \bar{P} n_i dS,
\label{displacement}
\end{equation}
%
or a Dirichlet condition, i.e., $u_i = u^{spec}_i$, on the displacement field. Above, $F^n_i$ is the force for
component $i$ due to a prescribed [static] pressure. 

\subsubsection{Intensity}
The boundary condition for each intensity assumes a grey, diffuse surface as, 
\begin{equation}
  I\left(s\right) = {1 \over \pi} \left[ \tau \sigma T_\infty^4 
                  + \epsilon \sigma T_w^4
                  + \left(1 - \epsilon - \tau \right) K \right].
\label{intBc}
\end{equation}

\subsection{Open Boundary Condition}
Open boundary conditions require far more care. In general,
open bcs are assembled by iterating faces and the boundary integration
points on the exposed face. The parent element is also required
since oftentimes gradients are used (for momentum). For an open boundary condition
the flow can either leave or enter the domain depending on what the computed mass
flow rate at the exposed boundary integration point is.

\subsubsection{Continuity}
For continuity, the boundary mass flow rate must also be computed. This value is stored and used
for the other equations that require advection. The same formula is used for the pressure-stabilized 
mass flow rate. However, the local pressure gradient
for each boundary contribution is based on the difference between the interior integration
point and the user-specified pressure which takes on the boundary value. The interior integration 
point is determined by linear interpolation. For CVFEM, full elemental averaging is used while in EBVC discretization,
the midpoint value between the nearest node and opposing node to the boundary integration point is used. In both
discretization approaches, non-orthogonal corrections are required. This procedure has been very important for 
stability for CVFEM tet-based meshes where a natural non-orthogonality exists between the boundary and
interior integration point.

\subsubsection{Momentum}
For momentum, the normal component of the stress is subtracted out we subtract out the normal component of the stress. The normal stress
component for component i can be written as $F_k n_k n_i$. The tangential 
component for component i is simply, $F_i - F_k n_k n_i$. As an example, the tangential
viscous stress for component x is,

\begin{equation}
  F^T_x = F_x - (F_x n_x + F_y n_y ) n_x,
\end{equation}
which can be written in general component form as,
\begin{equation}
  F^T_i = F_i(1-n_i n_i) - \sum_{i!=j} F_j n_i n_j.
\end{equation}

Finally, the normal stress contribution is applied based on the user specified pressure,
\begin{equation}
  F^N_i = P^{Spec} A_i.
\end{equation}

For CVFEM, the face gradient operators are used for the thermal stress terms. For EBVC discretization, 
from the boundary integration
point, the nearest node (the ``Right'' state) is used as well as the opposing node
(the ``Left'' state). The nearest node and opposing node are used to compute
gradients required for any derivatives. This equation follows the standard
gradient description in the diffusion section with non-orthogonal corrections used.
In this formulation, the area vector is taken to be the exposed area vector. 
Non-orthogonal terms are noted when the area vector and edge vector are not aligned.

For advection, If the flow is leaving the domain, we simply advect the nearest nodal value
to the boundary integration point. If the flow is coming into the domain,
we simply confine the flow to be normal to the open boundary integration 
point area vector. The value entrained can be the nearest node
or an upstream velocity value defined by the edge midpoint value. 

\subsubsection{Mixture Fraction, Enthalpy, Species, $k_{sgs}$, k and $\omega$ }
Open boundary conditions assume a zero normal gradient. When flow is entering the domain, the far-field
user supplied value is used. Far field values are used for property evaluations. When flow is leaving the domain, 
the flow is advected out consistent with the
choice of interior advection operator.

\subsection{Symmetry Boundary Condition}

\subsubsection{Continuity, Mixture Fraction, Enthalpy, Species, $k_{sgs}$, k and $\omega$}
Zero diffusion is applied at the symmetry bc.

\subsubsection{Momentum}
A symmetry boundary is one that is described by removal of the tangential stress. Therefore, only
the normal component of the stress is applied:

\begin{equation}
  F^n_x = (F_x n_x + F_y n_y ) n_x,
\end{equation}
which can be written in general component form as,
\begin{equation}
  F^n_i = F_j n_j n_i.
\end{equation}

\subsection{Periodic Boundary Condition}
A parallel multiple-periodic boundary condition is supported. Mappings are created between
master/slave surface node pairs. The node pairs are obtained from a parallel search and are expected
to be unique. The node pairs are used to map the slave global id to that of the master. This allows the linear
system to include matrix rows for only a subset of the overall set of nodes. Moreover, a periodic 
assembly for assembled quantities is managed via: $m+=s$ and $s=m$, where $m$ and $s$ are master/slave nodes, 
respectively. For each parallel assembled quantity, e.g., dual volume, turbulence quantities, etc., this procedure
is used. Periodic boxes and periodic couette and channel flow have been simulated in this code base. Tow forms of
parallel searches exist and are supported (one through the Boost TPL and another through the STK Search module).

\subsection{Non-conformal Boundary Condition}
A surface-based approach based on a DG method has been discussed in the 2010 CTR summer 
proceedings by Domino,~\cite{Domino:2010}. Both the edge- and element-based formulation 
currently exists in the code base using the CVFEM and EBVC approaches. 

Consider two domains, $A$ and $B$, which have a common interface, $\Gamma_{AB}$,
and a set of interfaces not in common, $\Gamma \backslash \Gamma_{AB}$
(see Figure~\ref{domainAB}), and assume that the solution of the 
time-dependent advection/diffusion equation is to be solved in both domains. 
Each domain has a set of outwardly pointing normals. In this cartoon, the interface 
is well resolved, although in practice this may not be the case. 


\begin{figure} 
  \centerline{\includegraphics[height=1.5in]{images/twoBlockDiag.pdf}} 
 \caption{Two-block example with one common surface, $\Gamma_{AB}$.} 
 \label{domainAB} 
\end{figure}    

An interior penalty approach, suitable for both the CVFEM and EBVC method, is constructed 
at each integration point at the exposed surface set. The numerical flux for a general 
scalar $\phi$ is constructed at the current integration point which is based on the current 
($A$) and opposing ($B$) elemental contributions,

\begin{equation} 
  \int \hat Q^A dS = \int [\frac{(q_j^A n_j^A - q_j^B n_j^B)}{2}
       + \lambda^A ( \phi^A - \phi^B) ]dS^A
       + \dot m^A \frac{(\phi^A + \phi^B)}{2} 
       + \eta \frac{|\dot{m}^A|}{2} (\phi^A - \phi^B),
\label{numericalFluxA}
\end{equation}
where $q_j^A$ and $q_j^B$ are the diffusive fluxes computed using the current 
and opposing elements and normals are outward facing. The penalty coefficient $\lambda^A$ contains the diffusive 
contributions averaged over the two elements,
\begin{equation} 
        \lambda^A = \frac{(\Gamma^A / L^A + \Gamma^B / L^B )}{2}.
\label{lamdbaA}
\end{equation}
Above, $\Gamma^k$ is the diffusive flux coefficient evaluated at current and opposing element location, respectively, 
and $L^k$ is an elemental length scale normal to the surface (again for current and opposing locations, $A$ and $B$). 
When upwinding is activated, the value of $\eta$ is unity. 

As written in Equation~\ref{numericalFluxA}, the default convection and diffusion term is a
Galerkin approach, i.e., equally averaged between the current and opposing face. The standard 
advection term is given by,
\begin{equation} 
        \int \rho \hat{u}_j \phi n_j dS.
\label{advection}
\end{equation}
For surface A, the form is as follows:
\begin{equation} 
        \int \rho \hat{u}_j^A \phi n_j^A dS^A = \dot m^A \frac{ \phi^A + \phi^B}{2},
\label{advection}
\end{equation}
with the nonconformal mass flow rate given by,
\begin{equation} 
        \dot {m}^A = [\frac{(\rho u_j^A + \gamma(\tau G_j^A p -\tau \frac{\partial p^A}{\partial x_j}))n_j^A
        				       - (\rho u_j^B + \gamma(\tau G_j^B p -\tau \frac{\partial p^B}{\partial x_j}))n_j^B}{2}
				       + \lambda^A ( p^A - p^B)] dS^A.
\label{mdotA}
\end{equation}
In the abve set of expressions, the consistent definition of $\hat{u}_j$, i.e., the convecting velocity including
possible pressure stabilization terms, is retained.

As with the interior advection scheme, the mass flow rate involves pressure stabilization terms. The value of 
$\gamma$ defines whether or not the full pressure stabilization terms are included in the mass flow rate expression.
Equation~\ref{mdotA} also forms the continuity nonconformal boundary contribution. 

With the substitution of $\eta$ to be unity, the effective convective term is as follows:
\begin{equation} 
        \int \rho \hat{u}_j \phi n_j^A dS^A = \frac{ (\dot m^A + |\dot m^A|) \phi^A +  (\dot m^A - |\dot m^A|)\phi^B}{2}.
\label{advectionAUPW}
\end{equation}
Note that this form reduces to a standard upwing operator.

Since this algorithm is a dual pass approach, a numerical flux can be written for the integration 
point on block $B$,
\begin{equation} 
        \int \hat Q^B dS = \int [\frac{(q_j^B n_j^B - q_j^A n_j^A)}{2}
				+ \lambda^B ( \phi^B - \phi^A) ]dS^A
        				+ \dot m^B \frac{(\phi^B + \phi^A)}{2} 
                                        + \eta \frac{|\dot{m}^B|}{2} (\phi^B - \phi^A).
\label{numericalFluxB}
\end{equation}
As with Equation~\ref{numericalFluxB}, $\dot{m}^B$ (see Equation~\ref{mdotB}) 
is of similar form to $\dot{m}^A$,

\begin{equation} 
        \dot {m}^B = [\frac{(\rho u_j^B + \gamma(\tau G_j^B p -\tau \frac{\partial p^B}{\partial x_j}))n_j^B
        				       - (\rho u_j^A + \gamma(\tau G_j^A p -\tau \frac{\partial p^A}{\partial x_j}))n_j^A}{2}
				       + \lambda^A ( p^B - p^A)] dS^B.
\label{mdotB}
\end{equation}

For low-order meshes with curved surface, faceting will occur. In this case, the outward facing normals may 
not be (sign)-unity factors of each other. In this case, it may be adventageous to define the opposing 
outward normal as, $n_j^B = -n_j^A$. 

Domino,~\cite{Domino:2010} provided an overview of a FEM fluids implementation. In such a formulation, the
interior penalty term appears, i.e.,

\begin{equation} 
   \int_{\Gamma_{AB}} \frac {\partial w^A}{\partial x_j} n_j \lambda (\phi^A-\phi^B) d\Gamma,
\end{equation}
and
\begin{equation} 
   \int_{\Gamma_{BA}} \frac {\partial w^B}{\partial x_j} n_j \lambda (\phi^B-\phi^A) d\Gamma.
\end{equation}
Although the sign of this term is often debated in the literature, the above set of expressions acts 
to increase penalty term stencil to include the full element contribution. 
As the CVFEM uses a piecewise-constant test function, this term is currently neglected.

Average fluxes are computed based on the current and opposing integration point locations. The 
appropriate DG terms are assembled as boundary conditions first with block $A$ integration 
points as $current$ (integrations points for block B are $opposing$) and then with block $B$ 
integration points as $current$ (surfaces for block A are, therefore, $opposing$). Figure~\ref{nonConformal} 
graphically demonstrates the procedure in which integration point values of the flux and penalty 
term are computed on the block $A$ surface and at the projected location of block $B$. 

\begin{figure}
\centering
  {\includegraphics[height=1.5in]{images/contactSearchAndEval.pdf}}
  \vspace{0.25in}
  \caption{Description of the numerical flux calculation for the DG algorithm. The 
    value of fluxes and penalty values on the current block ($A$) and the opposing block ($B$) are used 
    for the calculation of numerical fluxes. $\tilde \varphi$ represents the projected value.}
  \label{nonConformal}
\end{figure}

A parallel search is conducted to project the current integration point location to the opposing element exposed face. 
The search, therefore, provides the isoparametric coordinates on the opposing element. Elemental shape functions 
and shape function derivatives are used to construct the numerical flux for both the edge- and element-based 
scheme. The location of the Gauss points on the current element are either the Gauss Labatto or Gauss Legendre 
locations (input file specification). For each equation (momentum, continuity, enthalpy, etc.) the numerical 
flux is computed at each exposed non-conformal surface.

As noted, for most equations other than continuity and heat condition, the numerical flux includes advection and 
diffusion contributions. The diffusive contribution is easily provided using elemental shape function derivatives 
at the current and opposing surface. 

Above, special care is taken for the value of the mass flow rate at the non-conformal interface. Also,
note that the above written form does not upwind the advective flux, although the code allows for an upwinded 
approach. In general, the advective term contains contributions from both elements identified at the interface, 
specifically,

The penalty coefficient for the mass flow rate at the non-conformal boundary points is again a function of the 
blended inverse length scale at the current and opposing element surface location. The form of the mass flow 
rate above provides the continuity contribution and the form of the mass flow rate used in the scalar non-conformal
flux contribution.

The full connectivity for element integration and opposing elements is within the linear 
system. As such, for sliding mesh configurations, the linear system connectivity graph changes each time step. Recent prototyping of 
the dG-based and the overset scheme has allowed this method to be used across both disparate low-order 
topologies (see Figure~\ref{dgMixMatch}).

\begin{figure}[h!tp]
\centering
\includegraphics[clip,width=0.4\textwidth]{images/dgHex8Tet4Duct.png}
\includegraphics[clip,width=0.4\textwidth]{images/dgQuad4Quad9MMS.png}
\caption{Discontinuous Galerkin non-conformal interface mixed topology (left, hex8/tet4) and a low-order and high-order block interface 
(P=1 quad4 and P=2 quad9) for a MMS temperature solution (right). In the MMS image, the inset image is a close-up of the nodal 
Ids near the interface that highlights the quad4 and quad9 interface.}
\label{dgMixMatch}
\end{figure}
